\begin{abstract}
WebAssembly (Wasm) is a versatile binary instruction format enabling
high-performance code execution across diverse environments. As Wasm gains
momentum with continuous feature enhancements, the specification of its
execution semantics becomes pivotal.  The introduction of WebAssembly Domain
Specific Language ($\dsl$) has emerged as an innovative approach to streamline
the meticulous documentation process.  $\dsl$ serves as a front-end language
for defining semantics, and both formal and prose specifications are
automatically generated as back-ends.  While generating formal notations is
relatively straightforward, the challenge of generating accurate and consistent
prose notations arises due to the fundamental disparity between prose and
formal notations.

This paper introduces Algorithmic Language ($\al$), an executable language
designed to resemble prose specification, and presents an automated methodology
for extracting prose descriptions from $\dsl$. We identify challenges
during the process such as the "animation problem" and demonstrate how we
mitigate them. The extracted prose descriptions are rigorously tested against
the official WebAssembly test suite using an $\al$ interpreter, achieving a
100\% pass rate. \inred{This work not only addresses challenges in prose
notation for Wasm but also demonstrates the potential of $\al$ as a tool for
enhancing the precision and efficiency of programming language specifications
in general.}
\end{abstract}

