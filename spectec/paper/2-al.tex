\section{AL}\label{sec:al}

This section explains about syntax and semantics of AL.
\footnote{Note that this is NOT a syntax and semantics of WebAssembly, but rather,
syntax and semantics of prose notation of WebAssembly Specification}

\subsection{syntax}

This section describes the syntax of AL.

\begin{minipage}{0.5\textwidth}
$$
\begin{array}{@{}lrrl@{}}
\text{Program} & \mathsf{p} &::=& {\mathsf{A}^\ast} \\
\text{Algorithm} & \mathsf{A} &::=& \mathsf{algorithm}~\mathit{f}~({\mathit{e}^\ast})~\{{\mathsf{i}^\ast}\} \\
\text{Instruction} & \mathsf{i} &::=& \mathsf{if}~\mathit{c}~{\mathsf{i}^\ast}~{\mathsf{i}^\ast}^? \\ &&|&
\mathsf{either}~{\mathsf{i}^\ast}~{\mathsf{i}^\ast} \\ &&|&
\mathsf{assert}~\mathit{c} \\ &&|&
\mathsf{push}~\mathit{e} \\ &&|&
\mathsf{pop}~\mathit{e} \\ &&|&
\mathsf{popall}~\mathit{e}~\mathit{c} \\ &&|&
\mathsf{peek}~\mathit{e}~\mathit{c} \\ &&|&
\mathsf{let}~\mathit{e}~\mathit{e} \\ &&|&
\mathsf{call}~{\mathit{e}^?}~\mathit{f}~{\mathit{e}^\ast}~\{{({\mathit{x}^\ast} ; \mathit{iter})^\ast}\} \\ &&|&
\mathsf{trap} \\ &&|&
\mathsf{nop} \\ &&|&
\mathsf{return}~{\mathit{e}^?} \\ &&|&
\mathsf{execute}~\mathit{e} \\ &&|&
\mathsf{jumpin}~\mathit{e} \\ &&|&
\mathsf{jumpout} \\ &&|&
\mathsf{getfirst}~\mathit{e}~\mathit{c} \\ &&|&
\mathsf{replace}~\mathit{e}~\mathit{p}~\mathit{e} \\ &&|&
\mathsf{append}~\mathit{e}~\mathit{e} \\ &&|&
\mathsf{appendlist}~\mathit{e}~\mathit{e} \\
\text{Condition} & \mathit{c} &::=& \sim\mathit{c} \\ &&|&
\mathit{c}(\wedge)\mathit{c} \\ &&|&
\mathit{e}(<)\mathit{e} \\ &&|&
!\mathit{e} \\ &&|&
\mathit{e}<:\mathit{s} \\ &&|&
\mathsf{valid}~\mathit{e} \\
\end{array}
$$
\end{minipage}
\begin{minipage}{0.5\textwidth}
$$
\begin{array}{@{}lrrl@{}}
\text{Expression} & \mathit{e} &::=& \mathit{n} \\ &&|&
\mathit{s} \\ &&|&
\mathit{op}(\mathit{e^\ast}) \\ &&|&
\mathit{e}^\ast \\ &&|&
\mathit{e}^\mathit{e} \\ &&|&
\mathit{e}^? \\ &&|&
\mathit{e}::\mathit{e} \\ &&|&
|\mathit{e}| \\ &&|&
\{{(\mathit{s} \rightarrow \mathit{e})^\ast}\} \\ &&|&
\mathit{e}[\mathit{p}] \\ &&|&
\mathit{e}[\mathit{p}^\ast:+=\mathit{e}] \\ &&|&
\mathit{e}[\mathit{p}^\ast:=\mathit{e}] \\ &&|&
\mathit{s}(\mathit{e}^\ast) \\ &&|&
(\mathit{e},\mathit{e}) \\ &&|&
\mathit{x} \\ &&|&
\mathit{e}^{\{\mathit{x}^\ast\}~\mathit{iter}} \\
\text{Iter} & \mathit{iter} &::=& \mathsf{?} \\ &&|&
\mathsf{\ast} \\ &&|&
\mathsf{+} \\ &&|&
{(\mathit{x}<)}^?~\mathit{e} \\
\text{Path} & \mathit{p} &::=& [\mathit{e}] \\ &&|&
[\mathit{e}:\mathit{e}] \\ &&|&
.\mathit{s} \\
\text{Program} & \mathit{n} &\in& \mathbb{N} \\
\text{String} & \mathit{s} &\in& \mathit{Id} \\
\text{Variable name} & \mathit{x} &\in& \mathit{Id} \\
\text{Function name} & \mathit{f} &\in& \mathit{Id} \\
\end{array}
$$
\end{minipage}

Metavariable
$\mathsf{p}$ ranges over a program,
$\mathsf{A}$ ranges over an algorithm,
$\mathsf{i}$ ranges over an instruction,
$\mathit{c}$ ranges over a condition,
$\mathit{e}$ ranges over an expression,
$\mathit{iter}$ ranges over an iter-expression,
$\mathit{p}$ ranges over a path,
$\mathit{n}$ ranges over an integer,
$\mathit{s}$ ranges over a string,
$\mathit{x}$ ranges over a variable name.
and
$\mathit{f}$ ranges over a function name.

An AL program $\mathsf{p}$ consists of algorithms $\mathsf{A}^\ast$.  An
algorithm $\mathsf{A}$ has its name $\mathit{f}$, parameters
${\mathit{e}^\ast}$, and its body instructions ${\mathit{i}^\ast}$.  There are
various types of instructions including familiar push/pop instructions. The
instructions are designed in a way that it resembles the prose statements
appearing in the official specification.  For example, the prose `2. Pop the
value val from the stack` in Fig~\ref{fig:prosespec1} corresponds to the following
instruction: $\mathsf{pop} \mathit{val}$. There are also various kinds of
conditions and expressions.
\subsection{semantics}

This section describes the semantics of AL. It is based on an abstract machine
that models the program state, which consists of a stack and a store. An
algorithm is a entry point of an execution. One can execute an algorithm of the
program by calling it.  When an algorithm is called, the body instructions of
the algorithm are interpreted sequentially.

Interpreting one instruction may alter the internal program state of the
abstract machine, and the exact effect of each instruction is described using
semantics rule.

Figure ???  illustrate the rules for interpreting each
kind of instruction.
\inred{TODO: Write semantics of AL instructions in more detail. REQUIRED: Actually finish formalizing the semantics of AL.}

Conditions are calculated into a
boolean value, and the expressions are evaluated into a value.

Figure ??? illustrate the rules for calculating each kind of condition in
big-step semantics. \inred{TODO: Explain semantics of AL conditions in more detail.}

Figure ??? illustrate the rules for evaluating each kind of expression in big-step
semantics. \inred{TODO: Write semantics of AL expressions in more detail.}

Note that both calculating conditions and evaluating expressions are pure and
do not have any side-effect on the program state.


