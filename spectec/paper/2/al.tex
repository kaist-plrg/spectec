\section{Algorithmic Backends}\label{sec:al}
This section presents a mechanism for automatically generating
algorithmic representations from the formal semantics.
We define \emph{\dl}, a declarative language that defines the formal
semantics of Wasm (Sec.~\ref{sec:dl}),
\emph{\al}, an algorithmic language that defines the Wasm semantics
in a pseudocode style (Sec.~\ref{sec:aldef}),
and a \dl to \al translation (Sec.~\ref{sec:dl2al}).
We then describe how to generate prose specifications from the semantics
described in \al (Sec.~\ref{sec:prose}) and how to interpret the prose
specifications, which enables indirect interpretation of Wasm (Sec.~\ref{sec:interp}).

\subsection{\dl: Declarative Language}\label{sec:dl}
The DSL describes the formal semantics of Wasm, and
we abstract it into a \dl that only shows the features relevant to translation to an algorithmic representation.
Fig.~\ref{fig:dl-syntax} presents the syntax of \dl.

\begin{figure}[t]
\[
\small
\begin{array}{l@{~}c@{~}r@{~}l}
\text{Semantics} & \delta^*\\
\text{Definition} & \delta &::=& \rho\ \mid\ \lambda\\
\text{Reduction rule} & \rho &::=& \gamma \leadsto \gamma\ \mbox{---}\ \pi^*\\
\text{Configuration} & \gamma &::=&(\eta_\bot,\ \eta)\\
\text{Premise} & \pi &::=& \eta\ \mid\ \mathsf{\small otherwise}\\
\text{Helper function} &
\lambda &::=& \varx \kwrl \eta^* \kwrr\ \kwequ\ \eta\ \mbox{---}\ \pi^*\\
\text{Expression} & \eta &::=&
x\ \mid\ n\
\mid\ \kappa\ \eta^*\ \mid\ \eta + \eta\
\mid \eta^\eta\ \mid (\eta,\ \eta)\ \mid \eta^*
\mid \eta\ = \eta\ \mid \eta \leftarrow \eta\
\mid \cdots \\
\text{Constructor} & \kappa &::=&
\ssf{I32} \mid\ \ssf{CONST}\ \mid\ \ssf{REF.IS\_NULL}\ \mid\ \ssf{REF.NULL}\
\mid\ \ssf{LABEL\_}\
\mid\ \ssf{BR}\
\mid \cdots
\end{array}
\]
\caption{Syntax of \dl for Wasm}\label{fig:dl-syntax}
\end{figure}

The Wasm semantics is defined by a sequence of definitions $\delta^*$.
A definition is either a reduction rule $\rho$ or an auxiliary helper function $\lambda$.
A reduction rule $\gamma_1 \leadsto \gamma_2\ \mbox{---}\ \pi^*$ denotes that
when the current configuration of a program matches $\gamma_1$ and
all $\pi^*$ are satisfied, then the program configuration becomes $\gamma_2$.
A configuration $(\eta_\bot,\ \eta)$ denotes an optional Wasm program state $\eta_\bot$,
which is a pair of the current store and the current frame,
and a list of Wasm instructions $\eta$ that represents the current stack.
A premise $\pi$ is a condition expression $\eta$ or a special keyword
\ensuremath{\mathsf{\small otherwise}},
which denotes the negation of all the previous premises.
A helper function $\varx \kwrl \eta^* \kwrr\ \kwequ\ \eta\ \mbox{---}\ \pi^*$ denotes that
when a function named $\varx$ is called, its arguments are bound to parameters $\eta^*$
and a body expression $\eta$ is evaluated if all $\pi^*$ are satisfied.
An expression $\eta$ is a \dl expression.
Because the details of the expression is not relevant to this section,
we show only some cases used for concrete examples.
A constructor $\kappa$ is a string that denotes the name of
Wasm types, such as $\ssf{I32}$, or Wasm instructions, such as $\ssf{REF.IS\_NULL}$.

For example, the semantics of \inblue{\ensuremath{\mathsf{ref.is\_null}}}
in Fig.~\ref{fig:dsl} corresponds to the following in \dl:
\[
\begin{array}{l@{}l@{~}c@{~}l}
[&(\bot, [\upsilon, \ssf{REF.IS\_NULL}]) &\leadsto& (\bot, [\ssf{CONST\ I32\ 1}])\
\mbox{---}\ [\upsilon = \ssf{REF.NULL}\ t],\\
&(\bot, [\upsilon, \ssf{REF.IS\_NULL}])&\leadsto&(\bot, [\ssf{CONST\ I32\ 0}])\
\mbox{---}\ [\ssf{\small otherwise}] ]
\end{array}
\]
where a sequence is represented as comma separated elements
enclosed by $[$ and $]$ for readability.

\subsection{\al: Algorithmic Language}\label{sec:aldef}
\al is an algorithmic language that defines the Wasm language semantics in a pseudocode style.
Definitions in \dl are translated into definitions in \al,
so that algorithmic representations can be generated.

\begin{figure}[t]
\[
\small
\begin{array}{l@{~}c@{~}r@{~}lll}
\text{Program} & \prog &::=& \alg^*\\
\text{Algorithm} & \alg &::=&
\kwalg \; \varx \kwrl \expr^* \kwrr \; \kwcl \inst^* \kwcr \\
\text{Instruction} & \inst &::=&
    \kwif \; \cond \; \inst^* \; \inst^* & \mbox{If $\cond$, then: $\inst_1^*$ Else: $\inst_2^*$}\\
&& \mid&
  \kweither \; \inst^* \; \inst^* & \mbox{Either: $\inst_1^*$ Or: $\inst_2^*$}\\
&& \mid&
  \kwenter \; \expr \; \expr \; \inst^* & \mbox{Enter $\expr_1$ with $\expr_2$\ :\ $\inst^*$}\\
&& \mid&
  \kwassert \; \cond  & \mbox{Assert: Due to validation, $\cond$.}\\
&& \mid&
  \kwpush \; \expr  & \mbox{Push $\expr$ to the stack.}\\
&& \mid&
  \kwpop \; \expr  & \mbox{Pop $\expr$ from the stack.}\\
&& \mid&
  \kwpopall \; \expr  & \mbox{Pop all values $\expr$ from the stack.}\\
&& \mid&
  \kwlet \; \expr \; \expr & \mbox{Let $\expr_1$ be $\expr_2$.}\\
&& \mid&
  \kwtrap & \mbox{Trap.}\\
&& \mid&
  \kwnop & \mbox{Do nothing.}\\
&& \mid&
  \kwreturn \; \expr^? & \mbox{Return $\expr^?$.}\\
&& \mid&
  \kwexecute \; \expr & \mbox{Execute $\expr$.}\\
&& \mid&
  \kwexecuteseq \; \expr & \mbox{Execute the sequence $\expr$.}\\
&& \mid&
  \kwperform \; \varx \; \expr^* & \mbox{Perform $\varx(\expr^*)$.}\\
&& \mid&
  \kwexit & \mbox{Exit current context.}\\
&& \mid&
    \kwreplace \; \expr \; \qual \; \expr & \mbox{Replace $\expr_1[\qual]$ with $\expr_2$.}
\\

\text{Expression} & \expr &::=&
    \varx & \varx\\
&& \mid&
  \num & \num\\
&& \mid&
  \bcode{-} \; \expr & \bcode{-} \; \expr\\
&& \mid&
  \expr \; \binop \; \expr & \expr_1 \; \binop \; \expr_2\\
&& \mid&
  \expr \kwsl \qual \kwsr & \expr \kwsl \qual \kwsr\\
&& \mid&
  \expr \kwsl \qual \kwsr \; \kwass \; \expr & \mbox{$\expr_1$ with $\qual$ replaced by $\expr_2$}\\
&& \mid&
  \expr \kwsl \qual \kwsr \; \kwext \; \expr & \mbox{$\expr_1$ with $\qual$ appended by $\expr_2$}\\
&& \mid&
  \kwcl (\varx \mapsto \expr)^* \kwcr & \{\ (\varx:\expr)^*\ \}\\
&& \mid&
  \expr \; \kwcat \; \expr & \expr_1~\expr_2\\
&& \mid&
  | \expr | & \mbox{the length of $\expr$}\\
&&\mid&
  \cnstr \kwrl \expr^* \kwrr & \cnstr(\expr^*)\\
&& \mid&
  \varx \kwrl \expr^* \kwrr & \varx~\expr^*\\
&& \mid&
    \wasmc \\

\text{Condition} & \cond &::=&
\multicolumn{2}{l}{
    \kwnot \; \cond \mid
    \cond \; \binop \; \cond \mid
    \expr \; \binop \; \expr \mid
    \kwiscaseof \; \expr \; \varx \mid
    \wasmc
}
\\

\text{Path} & \qual &::=&
\multicolumn{2}{l}{
    \expr \mid
    \expr \bcode{:} \expr \mid
    \bcode{.} x
}
\\
%\text{Wasm} & \wasmcset &\ni& \wasmc \\
  % Wasm-specific condition
  % (* Conditions used in assertions *)
  % | TopLabelC                        (* "a label is now on the top of the stack" *)
  % | TopFrameC                       (* "a frame is now on the top of the stack" *)
  % | TopValueC of expr option (* "a value (of type expr)? is now on the top of the stack" *)
  % | TopValuesC of expr           (* "at least expr number of values on the top of the stack" *)
  \end{array}
\]
\vspace*{-1em}
\caption{Syntax of \al and its prose notation}\label{fig:al-syntax}
\end{figure}

Fig.~\ref{fig:al-syntax} presents the core syntax of \al.
The metavariables $\varx$ ranges over variables,
$\num$ ranges over numbers, and
$\cnstr$ ranges over constructors.
An \al program $\prog$ is a sequence of algorithms $\alg^*$,
which denotes the Wasm semantics in algorithmic form.
An algorithm $\alg$ consists of a name $\varx$, parameters~$\expr^*$,
and body instructions $\inst^*$. An algorithm denotes the semantics of a
certain Wasm instruction or an auxiliary helper function used to describe the language semantics.
An instruction~$\inst$ denotes a prose statement in the Wasm specification.
An expression $\expr$ denotes a prose expression that is evaluated to a value, and
a condition $\cond$ denotes a prose condition that is evaluated to a boolean value.
The figure also shows prose rendering of instructions and expressions.
The Wasm specification often uses specific phrases like ``the current frame'' and
``a label is now on the top of the stack.''
We abstract such Wasm-specific expressions and conditions as $\wasmc$ for brevity.

For example, the semantics of \inblue{\ensuremath{\mathsf{ref.is\_null}}}
in Fig.~\ref{fig:dsl} corresponds to the following in \al:
\[
\small
\!\!
\begin{array}{l}
\kwalg \; \mathit{REF.IS\_NULL} \kwrl \kwrr \; \kwcl \\
\qquad \kwassert \; \wasmc \\
\qquad \kwpop \; \mathit{val}\\
\qquad \kwif \; (\kwiscaseof \; \mathit{val} \; \mathit{REF.NULL})\;
\; (\kwpush \; \ssf{CONST}(\ssf{I32}(), 1)) \; (\kwpush \; \ssf{CONST}(\ssf{I32}(), 0))\\
\kwcr
\end{array}
\]

The semantics of $\al$ is conventional.
The interpretation of a program $\prog$ starts by calling one of its algorithms,
$\kwalg \; \varx \kwrl \expr^* \kwrr \; \kwcl \inst^* \kwcr$,
which sequentially executes its body instructions $\inst^*$.
Executing an instruction $\inst$ may alter the current program state,
and the resulting state after executing the last instruction of the algorithm
is the result of the program execution.
We define the semantics of $\al$ as a state transition system in a companiaon report~\cite{il-tr}.

\subsection{\dl to \al Translation}\label{sec:dl2al}
Now, we describe how to translate a Wasm semantics $\delta^*$ in \dl into an \al program $\prog$.
First, the definitions in $\delta^*$ are grouped to represent algorithms; among $\delta^*$,
the reduction rules $\gamma^*$ are grouped according to their target Wasm instructions, and
the helper functions $\lambda^*$ are grouped according to their names.
Each group is translated into a single algorithm in two phases:
1) preprocess the group's definitions to satisfy some \textit{preconditions}, and
2) generate an \al algorithm from the preprocessed \dl definitions.
Since the translation of helper functions is similar to the translation of reduction rules,
this section focuses on the translation of reduction rules.
A detailed description of the translation is available in the companion report~\cite{il-tr}.

\subsubsection{\dl to \dl Preprocessing}
Preprocessing consists of two steps: for each group of reduction rules
1) preprocess the left-hand sides of the reduction rules
in the group to make them the same, and
2) preprocess the premises of the definitions in the group
so that every variable is bound exactly once before its uses.

\medskip
The first preprocessing step is to unify the left-hand sides of
the reduction rules so that the left-hand sides in each group become identical.
Most Wasm definitions satisfy this precondition, but some do not, such as the following
DSL reduction rules for the \inblue{\ensuremath{\mathsf{br}}} instruction:

{
\begin{verbatim}
     rule Step_pure/br-zero:
        (LABEL_ n `{instr'*} val'* val^n (BR 0) instr*)  ~>  val^n instr'*

     rule Step_pure/br-succ:
        (LABEL_ n `{instr'*} val* (BR $(l+1)) instr*)  ~>  val* (BR l)
\end{verbatim}
}

\noindent
which corresponds to the following in \dl:
\[
\begin{array}{l@{~}l@{~}l@{~}l}
\rho_1 = &(\bot, (\ssf{LABEL\_} \; n \; x_i'^*\; x_v'^*\; x_v^{n}\; (\ssf{BR\; 0})\; x_i^*)) &\leadsto
 (\bot,\; x_v^{n}\; i^*)\ & \mbox{---}\ []\\
\rho_2 = &(\bot, (\ssf{LABEL\_} \; n \; x_i'^*\; x_v^*\; (\ssf{BR\; (}x_l\ssf{+1)})\; x_i^*)) &\leadsto
 (\bot,\; x_v^*\; (\ssf{BR}\; x_l))\ & \mbox{---}\ []
\end{array}
\]

For each group of reduction rules,
the unification algorithm \unify takes a list of left-hand side expressions as an input.
 For the \inblue{\ensuremath{\mathsf{br}}} instruction, for example,
 \unify takes two expressions:
\[
\begin{array}{l}
\eta_1 = (\ssf{LABEL\_} \; n \; x_i'^*\; x_v'^*\; x_v^{n}\; (\ssf{BR\; 0})\; x_i^*)\\
\eta_2 = (\ssf{LABEL\_} \; n \; x_i'^*\; x_v^*\; (\ssf{BR\; (}x_l\ssf{+1)})\; x_i^*)
\end{array}
\]

For a list of expressions to unify $\eta_1\ \cdots\ \eta_n$,
\unify 1) generates a unified expression $\eta$ possibly containing some fresh variables
and then 2) generates premises $\pi_i^*\ (1\le i \le n)$ using the fresh variables to make
each $\eta_i$ be the same as $\eta$ with $\pi_i^*$.
More specifically, the unified expression~$\eta$ is the most common expression of
the expressions to unify $\eta_1\ \cdots\ \eta_n$,
replacing the different components with fresh variables.
For example, the unified expression for the reduction rules for \inblue{\ensuremath{\mathsf{br}}} is
{$(\ssf{LABEL\_} \; n \; x_i'^*\; y\; (\ssf{BR}\; y')\; x_i^*$)} with two fresh variables $y$ and $y'$.
After generating the unified expression, \unify generates premises $\pi_i^*$
for each $\eta_i$ so that $\eta_i$ is an instance of the unified expression satisfying $\pi_i^*$.
For example, to make the unified expression for \inblue{\ensuremath{\mathsf{br}}}
same as the left-hand sides of $\rho_1$ and $\rho_2$, \unify infers the conditions
$[y = x_v'^*\; x_v^{n},\ y' = 0]$ and $[y = x_v^*,\ y' = x_l\ssf{+1}]$, respectively.
Finally, the unification result of the rules for the \inblue{\ensuremath{\mathsf{br}}} instruction is as follows:
\[
\begin{array}{l@{~}l@{~}l@{~}l}
\rho_1 = &(\bot, (\ssf{LABEL\_} \; n) \; x_i'^*\; y\; (\ssf{BR}\; y')\; x_i^*) &\leadsto
 (\bot, v^n\; x_i^*)\ & \mbox{---}\ [y = x_v'^*\; x_v^{n},\ y' = 0]\\
\rho_2 = &(\bot, (\ssf{LABEL\_} \; n) \; x_i'^*\; y\; (\ssf{BR}\; y')\; x_i^*) &\leadsto
 (\bot, x_v^*\; (\ssf{BR}\; x_l))\ & \mbox{---}\ [y = x_v^*,\ y' = x_l\ssf{+1}]
\end{array}
\]
Thanks to the fresh variables $y$ and $y'$, both rules $\rho_1$ and $\rho_2$ now have
the same left-hand sides. The definitions of $y$ and $y'$ are available as premises:
$[y = x_v'^*\; x_v^{n},\ y' = 0]$ for $\rho_1$ and
$[y = x_v^*,\ y' = x_l\ssf{+1}]$ for~$\rho_2$.


\medskip
The second preprocessing step is to change the premises of each group's definitions
so that each variable is bound exactly once before it is used.
This step is necessary because the order of premises in declarative reduction rules
can be arbitrary.
In addition, equality expressions in premises can be ambiguous because they can represent equality check conditions or bidirectional variable bindings.
This second preprocessing step replaces every equality expression denoting a variable binding with $\eta \leftarrow \eta'$.
Thus, this step identifies each variable's binding occurrence,
keeps track of the variables that each premise binds,
and reorders premises so that preceding premises bind all free variables in each premise.

For example, consider the first rule for the \inblue{\ensuremath{\mathsf{br}}} instruction again:
\[
\begin{array}{l@{~}l@{~}c@{~}l}
\rho_1 = &(\bot, (\ssf{LABEL\_} \; n \; x_i'^*\; y\; (\ssf{BR}\; y')\; x_i^*)) &\leadsto&
 (\bot, x_v^{n}\; x_i^*)\ \mbox{---}\ [y = x_v'^*\; x_v^{n},\ y' = 0]\\
\end{array}
\]
The second premise $y' = 0$ is an equality check codition,
while the first premise is a binding of fresh variables $x_v$ and $x_v'$.
Therefore, this step changes the rule as follows:
\[
\begin{array}{l@{~}l@{~}c@{~}l}
\rho_1 = &(\bot, (\ssf{LABEL\_} \; n \; x_i'^*\; y\; (\ssf{BR}\; y')\; x_i^*)) &\leadsto&
 (\bot, x_v^{n}\; x_i^*)\ \mbox{---}\ [x_v'^*\; x_v^{n} \leftarrow y,\ y' = 0]\\
\end{array}
\]
which clearly indicates that fresh variables $x_v$ and $x_v'$ are newly introduced in the first premise.
Once the variable binding occurrences in premises are identified,
reordering the premises is a simple def-use dataflow analysis.

The primary challange now lies in identifying each variable's binding occurrence in premises.
An interesting complication arises due to the fact of a \textit{parital
binding}, where, among the free variables appearing on one side of an equality
expression, some are identified as binding occurrences, while others are not.
For example, when we have the premise $(x_1, x_2) = y$, it is possible to
identify only x_1 as binding occurence but not $x_2$.
Unfortunately, such partial bindings incurs the introduction
of fresh variables when generating prose. In the aformentioned exmaple,
the premise wwould to a rendeing like ``1. Let ($x_1$, $t$) be $y$; 2. If $t$ =
$x_2$, then: ...'', introducing a new free variable $t$.  The excessive use of such
fresh variables could result in a prose specification that deviates significantly from the
current prose specification, eventaully leading to a less readable document.

Therefore, it is desirable to minimize the number of partial bindings when
identifying the binding occurences, which is not a trivial task.
In fact, this turns out to be a \textit{NP-hard problem}.
We prove it by reduction from a known NP-hard problem, the \textit{exact cover} problem~\cite{exactcover}:
\begin{quote}
The exact cover problem aims at deciding whether it is possible to select some subsets within a given collection of subsets in such a way that each element of a given set belongs to exactly one selected subset. This problem is NP-complete~\cite{karp72}.
\end{quote}
Here, we provide the formal definition of the exact cover problem:
\begin{definition}[Exact Cover Problem]\label{def:exactcover}
An instance of the \textit{Exact Cover Problem} (EC) is defined by a tuple $(X, S)$
such that $X$ is a set of elements and $S \subseteq \mathcal{P}(X)$ is a collection of subsets of $X$.
EC aims at deciding if there exists a subcollection $P \subseteq S$ which is a partition of $X$,
that is, $\forall x \in X.\ |\{ s \in P \mid x \in s \}| = 1$.
\end{definition}

\noindent
For example, consider $X = \{a, b, c, d, e\}$ and $S = \{\{a,b\}, \{b,c\}, \{c,d,e\}\}$.
Because $\{\{a,b\}, \{c,d,e\}\}$, one of the subcollections of $S$, is a partition of $X$,
the answer is yes.
On the other hand, for $S' = \{\{a,b\}, \{b,c\}, \{c,d\}, \{d,e\}\}$,
since no subcollection of $S'$ is a partition of $X$\footnote{It can be easily
verified by the parity. The union of any pair-wise disjoint subset of $S'$ would have an even number
of elements, but the whole set $X$ has 5 elements.}, the answer is no.

Because EC is one of Karp's 21 NP-complete problems~\cite{karp72},
we can prove that the problem of identifying each variable's binding occurrence in premises is NP-hard,
if we can reduce EC into the problem in polynomial time.

\begin{theorem}\label{thm:np-hard}
The problem of identifying each variable's binding occurrence in premises with minimal partial binding is NP-hard.
\end{theorem}
\begin{proof}
Assume that we are given EC with a set $X$ and
a collection of its subsets $S \subseteq \mathcal{P}(X)$. Let $n$ be the size of $S$.
Let $S_i = \{x_{i1}, x_{i2}, ..., x_{ij}\}$ be the $i$-th subset of $S$.
Now, consider a reduction rule $\gamma \leadsto \gamma'\ \mbox{---}\ \pi^*$
where $\gamma = (\bot,\ v_n, v_{n-1}, \cdots, v_1)$,
$\gamma' = (\bot,\ [])$, and
the $i$-th premise of $\pi^*$ be $v_i = (x_{i1}, x_{i2}, ..., x_{ij})$.
Note that this reduction rule can be constructed in linear time.
The claim is that when we identify each variable's binding occurrence in $\pi^*$
with the minimal number of partiall binding,
this gives a solution to EC.
Specifically, the answer to EC is YES if and only if the if the minimum nuber is zero.
Note that every variable must be bound exactly once.
Thus, given that there is no partial binding,
when we collect all premises that bind new variables,
then the subsets $S_i$ corresponding to such premises
would form a subcollection of $S$, which is a partition of $X$.
Conversely, if an exact cover exists for the given collection, then the
identification of variable's binding occurence in premises can be done
without any partial bindings, by regarding every variables in the
premises that corresponds to the sets in the exact cover as binding
occurence, and regarding anything else as not.
\end{proof}

\textbf{Example.}
Let's consider the example of $X = \{a, b, c, d, e\}$ and $S = \{\{a,b\}, \{b,c\}, \{c,d,e\}\}$ again.
Following the proof above, EC for $X$ and $S$ is reduced to the problem of
identifying each variable's binding occurrence in the premises of the
following reduction rule:
\[
\begin{array}{l@{~}l@{~}c@{~}l}
\rho = & (\bot, [v_3, v_2, v_1]) &\leadsto& (\bot, [])\ \mbox{---}\
[v_1 = (a,b), v_2 = (b,c), v_3 = (c, d, e)]
\end{array}
 \]
If we successfully solve the problem, then the result should look like the following:
\[
\begin{array}{l@{~}l@{~}c@{~}l}
\rho' = & (\bot, [v_3, v_2, v_1]) &\leadsto& (\bot, [])\ \mbox{---}\
[(a,b) \leftarrow v_1, (c, d, e) \leftarrow v_3, v_2 = (b,c)]
\end{array}
\]
which does not have any partial bindings. From the result, we can reconstruct the partition of the set $X$
by collecting the binding premises, $\{\{a, b\}, \{c, d, e\}\}$, giving the answer to EC.

\begin{algorithm}[t]
\DontPrintSemicolon
\KwIn{A list of premises $\pi^*$ and a set of bound variables $V$}
\KwOut{A list of preprocessed premises $\pi'^*$}
$S \gets \emptyset$\;
\For{$i \gets 1$ \textbf{to} $|\pi^*|$}{
    \lIf{$\pi_i = (\eta = \eta')$}
    {$S_i \gets \{\{i\},\ \{i\}\cup\mathit{free}(\eta),\ \{i\}\cup\mathit{free}(\eta')\}$}
    \lElse{$S_i \gets \{\{i\}\}$}
$S \gets S \cup S_i$\;
}
$P \gets \mathit{Knuth}(S \cup \{X\})$\;
\For{$p \in P$}{
    $\pi'^* \gets \mathit{replaceVariableBinding}(\pi^*, p)$\;
    \lIf{reordering $\pi'^*$ succeeds}
    {\Return{reordered $\pi'^*$}}
}
where $\mathit{free}(\eta)$ returns free variables in $\eta$ and
$\mathit{Knuth}(S)$ returns partitions of $\bigcup S$
\caption{Preprocess Premises}
\label{algo:preminfer}
\end{algorithm}

\medskip
Thus, no polynomial-time algorithm can solve the problem.
If the numbers of premises are small, a simple brute-force algorithm might be a solution.
However, the reduction rule for module instantiation, for example,
has more than 10 premises and variables, so a more efficient method is required.
Another solution is to use an SMT solver like Z3~\cite{z3},
since this problem can be treated as a constraint solving problem.
However, the performance overhead of Z3 does not apply to this problem either.

As a practical solution to this NP-hard problem,
we adopt an all-or-nothing heuristic approach.
We first try to solve the problem under a condition that the partial bindings are not allowed,
and if that fails, we resort to a greedy algorithm, inevitably introducing some fresh variables.
In order to solve the problem with the constraint of no-partial bindings,
we reduce the problem into EC\footnote{Note that the direction is opposite with
the proof} and then adopt the Knuth algorithm~\cite{knuth2000dancing},
a well-known and effective algorithm for solving EC.
The high-level idea is to encode the premises as a collection of sets,
where a solution to EC of the encoded collection corresponds to the solution to our problem.

Algorithm~\ref{algo:preminfer} describes the process.
It takes two inputs, a list of premises $\pi^*$ in reduction rules
and a set of already bound variables $V$, and returns a list of new premises $\pi'^*$.
First, it encodes premises as a collection of subsets of $X$,
where $X = \{1, \cdots, |\pi^*|\} \cup \mathit{free}(\pi^*)$
is a set of the numbers from $1$ to the size of premises
and all free variables in them, on which the Knuth algorithm performs.
For each $\pi_i$, if it is an equality expression $\eta = \eta'$,
it is encoded as three subsets: $\{\{i\},\ \{i\}\cup\mathit{free}(\eta),\ \{i\}\cup\mathit{free}(\eta')\}$,
which denotes its possible interpretations.
The first set $\{i\}$ denotes when the $i$-th premise does not bind any variables,
meaning $\pi_i$ is an equality check condition.
The second set $\{i\} \cup \mathit{free}(\eta)$ denotes when $\pi_i$ binds ALL the variables in $\eta$ and
the third set $\{i\} \cup \mathit{free}(\eta')$ denotes when $\pi_i$ binds ALL the variables in $\eta'$.
For example, if the first premise is $x = y$, it is encoded as three subsets: $\{1\}$, $\{1, x\}$, and $\{1, y\}$.
If $\pi_i$ is not an equality expression, it is encoded as only one subset, $\{i\}$,
meaning $\pi_i$ does not bind any variables but checks some non-equality condition.
Then, the Knuth algorithm takes the collection $S$ containing all the encoded subsets and
the set of bound variables $V$ and returns the partitions $P$ of the set $X$.
Note that the Knuth algorithm may not return a unique partion.
In addition, due to the definition of partition, for any partion $p$,
there should be exactly one subset that contains a number $i$ for $1 \le i \le n$.
Thanks to the design of encoding, a subset is either a singleton set $\{i\}$ or
a set with an index and some variables $\{i, x_1, x_2, \cdots\}$.

\begin{algorithm}[t]
\DontPrintSemicolon
\KwIn{A list of premises $\pi^*$ and a partition $p$}
\KwOut{A list of premises $\pi'^*$ with explicit variable binding }
\SetKwProg{Fn}{Function}{}{}
\Fn{$\mathit{replaceVariableBinding}(\pi^*, p)$} {
\For{$i \gets 1$ \textbf{to} $|\pi^*|$}{
\If{$\pi_i = (\eta = \eta')$}
    {\lIf{$\{i\} \in p$}{$\pi'_i \leftarrow (\eta = \eta')$}
     \lElseIf{$\{i\}\cup\mathit{free}(\eta) \in p$}{$\pi'_i \leftarrow (\eta \leftarrow \eta')$}
     \lElseIf{$\{i\}\cup\mathit{free}(\eta') \in p$}{$\pi'_i \leftarrow (\eta' \leftarrow \eta)$}
    }
\lElse{$\pi'_i \leftarrow \pi_i$}
}
\Return{$\pi'^*$}
}
\caption{Replace Variable Binding}
\label{algo:binding}
\end{algorithm}

Using a partition $p$, it replaces every equality expression denoting a variable binding
with an explicit variable binding expression via $\mathit{replaceVariableBinding}$ in Algorithm~\ref{algo:binding}.
For example, if a singleton set $\{2\}$ is in the input partition $p$,
then the second premise $\pi_2$ is a condition and therefore remains the same.
If a set $\{3, y\}$ is in $p$ and the third premise is $x = y$,
then the third premise is replaced with $y \leftarrow x$.
For preprocessed premises $\pi'^*$, Algorithm~\ref{algo:preminfer} tries to reorder them
so that all variables are bound before their uses.
Reordering premises may fail, if they contain cyclic bindings like $x \leftarrow f(y)$ and $y \leftarrow g(x)$.
If reordering $\pi'^*$ succeeds, the algorithm returns the reordered $\pi'^*$;
otherwise, it tries with the next partition.
If the reordering fails for every possible partition, the translation from \dl to \al fails.
In such cases, the \dl definitions should be rewritten to remove cyclic bindings in premises.


\subsubsection{\al Generation}
After preprocessing, \dl definitions satisfy two preconditions:
1) the left-hand sides of the reduction rules in each group
are identical and 2) every variable in premises is bound exactly once before its uses.
For each group of reduction rules $\rho^* =
(\eta_1, \eta' _1)  \leadsto \gamma_1\ \mbox{---}\ \pi_1^*\
\cdots\
(\eta_n, \eta' _n) \leadsto \gamma_n\ \mbox{---}\ \pi_n^*$,
the translation algorithm \dltoil generates an \al algorithm
by translating the left-hand-side $(\eta_1, \eta' _1)$ first,
and then translating the premise $\pi_i$ and right-hand-side $\gamma_i$
for each rule $\rho_i (1\le i\le n)$.
For the rules for \inblue{\ensuremath{\mathsf{ref.is\_null}}} in \dl, for example:
\[
\begin{array}{l@{~}l@{~}c@{~}l}
\rho_1 = &(\bot, [\upsilon, \ssf{REF.IS\_NULL}]) &\leadsto& (\bot, [\ssf{CONST\ I32\ 1}])\
\mbox{---}\ [\upsilon = \ssf{REF.NULL}\ t]\\
\rho_2 = &(\bot, [\upsilon, \ssf{REF.IS\_NULL}])&\leadsto&(\bot, [\ssf{CONST\ I32\ 0}])\
\mbox{---}\ [\ssf{\small otherwise}]
\end{array}
\]
the identical left-hand-side $(\bot, [\upsilon, \ssf{REF.IS\_NULL}])$
is translated to $[\kwassert \; \wasmc, \kwpop \; \mathit{val}]$.
For the first rule $\rho_1$, the premise ``$\upsilon = \ssf{REF.NULL}\ t$'' and
the right-hand-side ``$(\bot, [\ssf{CONST\ I32\ 1}])$'' are translated to
``$\kwif \; (\kwiscaseof \; \mathit{val} \; \mathit{REF.NULL})$'' and
``$\kwpush \; \ssf{CONST}(\ssf{I32}(), 1)$,'' respectively.

Roughly speaking, translating a left-hand-side of a reduction rule corresponds to
generating the beginning of an algorithm, which pops the values from the stack
to use as the inputs for the target instruction and binds new variables
that contain the information about the inputs.
Translation of premises corresponds to generation of the middle of an algorithm,
which generates either variable bindings $\kwlet \; \expr \; \expr$
or $\kwif$ instructions with conditions $\cond$.
Note that a binding expression $\eta \leftarrow \eta'$ may introduce side conditions.
For example, a binding $[x, y] \leftarrow \mathit{arr}$ introduces a side condition
that the size of the array $\mathit{arr}$ is two.
\dltoil generates such side conditions based on binding patterns.
Finally, translation of a right-hand-side of a reduction rule corresponds to
generation of the end of an algorithm, which pushes the result value onto the stack top
or execute other Wasm instructions.
Due to space limitation, we refer the interested readers to a companion report~\cite{il-tr}.
