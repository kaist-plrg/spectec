\section{Evaluation}
\label{sec:eval}
We developed \dslname as an open-source project~\cite{spectec}, and
evaluated it based on the following research questions:
\begin{itemize}
\item \textbf{RQ1. Correctness}:
Does \dslname correctly generate formal and prose specifications, as well as the interpreter backend?
(Sec.~\ref{sec:correctness})
\item \textbf{RQ2. Bug Prevention}:
Can \dslname detect or prevent bugs during Wasm standard development?
(Sec.~\ref{sec:bug})
\item \textbf{RQ3. Forward Compatibility}:
Can \dslname support future language features with minor modifications?
(Sec.~\ref{sec:forward})
\end{itemize}

We evaluate \dslname with the latest Wasm specification, Wasm 2.0~\cite{wasmspec};
we manually specified its syntax, validation, and execution semantics
in the DSL, which we name \specdsl. We do not specify the SIMD
instructions and auxiliary functions for numeric instructions
since they are numerous and do not make new challenges for \dslname.
Note that we specify some definitions differently from the Wasm 2.0 standard.
For example, while Wasm 2.0 uses evaluation contexts to escape multiple block contexts
simultaneously, \specdsl uses a bubbling-up technique to escape one context at a time.
%One special reduction rule eval_expr, which is specifying the evaluation of Wasm expression, is treated in ad-hoc manner in the current specification. Despite being a reduction rule, the prose for this reduction rule is written as if it were an auxiliary helper function. Following this convention in the actual spec, we manually hardcoded an algorithmic prose for this rule.
We also rewrite the module instantiation semantics, \textit{instantiate},
to remove cyclic bindings within premises.
\specdsl amounts to \inred{1,147} Lines of Code (LoC),
while the corresponding ReStructuredText document for Wasm 2.0 is
\inred{ten thousand} LoC.

\subsection{Correctness}\label{sec:correctness}
We evaluate the correctness of the artifacts generated by \dslname using \specdsl as an input.

\subsubsection*{Formal and Prose Specification}
\dslname can generate two of the four key artifacts required by the W3C Wasm
Community Group in order to standardize a feature.
It can generate a formal specification in declarative-style rewrite rules, written in LaTeX,
for the syntax, validation, and execution semantics.
It can also generate a prose pseudocode presenting an algorithmic-style semantics,
written in reStructuredText markup, for the execution semantics.
A generated PDF document with formal and prose
notations~\cite{specdsl}\footnote{Due to double-blind reviewing,
we submit it as a supplementary material.} is very close to the
handwritten specification~\cite{wasmspec}.

\subsubsection*{Interpreter Backend}
\dslname allows indirect execution of Wasm programs using the reference interpreter's parser.
It also uses the reference interpreter's implementations for numeric functions.
We evaluate its correctness by executing the official Wasm test suite~\cite{wasmtest}.
This evaluation demonstrates two things: 1) that the meta-level interpreter can interpret
Wasm programs; and 2) that \specdsl, the process of translating it to \al, and
the resultant Wasm semantics in \al are correct.
A Wasm test consists of one or more Wasm modules and
assertions to verify whether the implementation behaves as expected.
Of the seven kinds of assertions~\cite{wast},
we exclude three related to parsing and validation
(for which \dslname does not support algorithmic backends)
and one related to testing infinite loops.
Thus, we use the following three kinds of assertions:

{\small
\begin{verbatim}
  (assert_return <action> <result>*) ;; assert action has expected results
  (assert_trap <action> <failure>)   ;; assert action traps with given failure string
  (assert_trap <module> <failure>)   ;; assert module traps on instantiation
\end{verbatim}
}

\noindent
We performed our experiments with an Ubuntu machine with a 4.0GHz
Intel(R) 390 Core(TM) i7-6700k and 32GB of RAM (Samsung DDR4 2133MHz 391 8GB*4).
On this machine, the meta-level interpreter passed all \inred{23,751} tests
(\inred{21,363} \stt{assert\_return},
\inred{2,354} \stt{assert\_trap} for actions, and
\inred{34} \stt{assert\_trap} for modules)
in \inred{21.349} seconds.
Every applicable test in the official Wasm test suite has successfully passed,
implying the correctness of \dslname.

\subsection{Bug Prevention}\label{sec:bug}
We evaluate \dslname's ability to detect or prevent bugs
during the Wasm standard development process
by checking whether it can prevent
the actual bugs that occurred during the standard's development~\cite{wasmspecrepo}.
We classify the bugs into four groups.

\subsubsection*{Type Errors}
We found two bugs that \dslname's type checking may have prevented:
1) a missing type in the execution semantics of the \inblue{\ensuremath{\mathsf{elem.drop}}}
instruction\footnote{\url{https://github.com/WebAssembly/spec/commit/5b18d52}} and
2) a missing argument in the execution semantics of module
instantiation\footnote{\url{https://github.com/WebAssembly/spec/commit/793b3ff}}.
We injected each bug into \specdsl, used the revised DSL with a bug as an input to \dslname,
and confirmed that \dslname detected both bugs as type errors with 
informative error messages that included the error locations in the specification
as well as the reasons for the errors.

\subsubsection*{Prose Errors}
We found five bugs that \dslname's automatic prose generation may have prevented:
1) free identifiers in the execution semantics of control
instructions\footnote{\url{https://github.com/WebAssembly/spec/commit/be820b2}},
2) a free identifier in the execution semantics of the \inblue{\ensuremath{\mathsf{alloc\_elem}}} 
instruction\footnote{\url{https://github.com/WebAssembly/spec/commit/04beeb7}},
3) free identifiers in the execution semantics of the \inblue{\ensuremath{\mathsf{memory.init}}} 
instruction\footnote{\url{https://github.com/WebAssembly/spec/commit/4353b29}},
4) a missing argument in the execution semantics of the \inblue{\ensuremath{\mathsf{table.set}}} 
instruction\footnote{\url{https://github.com/WebAssembly/spec/commit/f6ae547}}, and
5) an obsolete step in the execution semantics of function
calls\footnote{\url{https://github.com/WebAssembly/spec/commit/f54b5b8}}.
Such prose errors do not exist in prose specifications
generated by \dslname, and we confirmed that the corresponding semantics
were correctly specified in the generated PDF document~\cite{specdsl}.

\subsubsection*{Semantics Errors}
We found three bugs that \dslname's meta-level interpretation may have prevented:
1) a missing value in the value stack of the reduction rule of the \inblue{\ensuremath{\mathsf{table.grow}}}
instruction\footnote{\url{https://github.com/WebAssembly/spec/commit/3545ad0}},
2) a wrong memory index for the \inblue{\ensuremath{\mathsf{memory.fill}}}
and \inblue{\ensuremath{\mathsf{memory.init}}}
instructions\footnote{\url{https://github.com/WebAssembly/spec/commit/e7f6e1c}}, and
3) popping a wrong number of values when exiting from
a label\footnote{\url{https://github.com/WebAssembly/spec/commit/8f5c489}}.
We injected each bug into \specdsl, used the revised DSL with a bug as an input to \dslname,
executed the Wasm test suite with the generated meta-level interpreter,
and confirmed that the meta-level interpreter detected all three bugs as semantics errors.

\subsubsection*{Editorial Fixes}
Finally, we found numerous editorial fixes that address presentational issues
such as typographical errors in LaTeX or inconsistencies in writing style across the specification.
These errors do not exist in formal and prose specifications generated by \dslname
since they follow a predefined structure and style,
removing the possibility of human errors or inconsistencies.

\medskip
These results demonstrate that \dslname is effective in preventing a
wide range of human errors throughout the Wasm standardization process.
It can detect or prevent various errors, including type errors, prose errors,
semantics errors, and editorial fixes.
We believe that \dslname can significantly enhance the robustness and
reliability of the Wasm standardization process
while also reducing the burden on specification authors.

\subsection{Forward Compatibility}\label{sec:forward}
We evaluate \dslname's forward compatibility by applying it to
five proposals ready for inclusin in Wasm 3.0:
Garbage Collected Types~\cite{wasm-gc},
Tail Call Extension~\cite{wasm-tce},
Function Reference Types~\cite{wasm-frt},
Multiple Memories~\cite{wasm-mm}, and
Extended Constant Expressions~\cite{wasm-ece}.
The proposals introduce or modify 34 Wasm instructions.

For each proposal, we extended \specdsl with the proposal described in the DSL.
Describing all five proposals required only minor adjustments in the DSL,
mostly adding new custom operators for proposal-specific notations.
Then, we utilized \dslname to generate formal and prose specifications automatically.
\dslname generated formal specifications for all 34 Wasm instructions and
prose specifications for 32 of them,
but failed to generate prose specifications for two of them. 
The remaining two instructions' reduction rules were not translated to \al
since the \dl to \al translator does not allow their writing style, such as inverse functions. 
\dslname correctly generated their prose specifications
after we revised the rules to reflect the translator's intended style.
\dslname correctly generated their prose specifications.
Extending the translator's capability to handle reduction rules with diverse styles is our future work.
The proposals also introduce dozens of new auxiliary helper
functions, all of which were successfully translated.
Finally, we executed the proposals' tests using the generated meta-level interpreter.
For the subtyping validation of Wasm types introduced by the Garbage Collected Types proposal,
\dslname uses the reference interpreter's implementation.
During this evaluation, we found \inred{nine} bugs in the proposals:
two type errors, one prose error, four semantics errors, and two editorial fixes.
We reported them to the specification authors and received conformation~\cite{proposalbugs}.
After fixing the errors in the proposals, we applied \dslname to the revised \specdsl.
For each proposal, the meta-level interpreter passed all the tests.

The results demonstrate that, with little adjustments,
\dslname can handle future language features and detect and prevent
numerous human mistakes throughout the standardization process.
We believe that \dslname is a long-term solution for supporting a growing
language like Wasm.
