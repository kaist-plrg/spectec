\subsection{Interpreter Backend}\label{sec:interp} %1p
\dslname supports the interpretation of Wasm programs by \textit{meta-level interpretation}.
By interpreting an \al program that denotes the Wasm semantics, with
a Wasm program as its input value, we can indirectly interpret the Wasm program.

This approach was previously used by the ESMeta framework~\cite{esmeta,jiset}.
ESMeta extracts two main components from the ECMAScript specification~\cite{ecmascript}:
a parser and executable semantics.
The parser is generated from the grammar section in the specification,
which takes a JavaScript program as input and returns a parsed AST as output.
The executable semantics is generated from the structured English prose algorithms in the specification,
and is translated into an internal representation, \ires.
\ires has its own semantics, so it can be executed by an interpreter implementation.
By executing the extracted semantics written in \ires using the parsed AST as input to the semantics,
we can indirectly execute JavaScript programs.

\dslname allows indirect execution of Wasm programs in a similar way.
A parser parses a Wasm program into an AST, which an \al algorithm takes as input.
Unlike ESMeta, \dslname does not generate a Wasm parser,
but uses the parser of the reference interpreter~\cite{wasmparser}.
The executable semantics written in \al is automaitcally extractd as described in Sec.~\ref{sec:dl2al}.
Based on the \al semantics~\cite{il-tr}, we developed an \al interpreter in OCaml.
By executing the extracted semantics written in \al using the parsed AST as input to the semantics,
we can indirectly execute Wasm programs.

As described in Sec.~\ref{sec:aldef}, we can execute an \al program by calling one of its algorithms.
A Wasm program is executed when instantiating a module~\cite[Sec. 4.5.4]{wasmspec} or
invoking a function on a module instance~\cite[Sec. 4.5.5]{wasmspec}.
Thus, the \al interpreter calls the $\mathit{instantiate}$ algorithm to instantiate a module
or the $\mathit{invoke}$ algorithm to invoke one of the functions within the module.
After calling the algorithm, it returns a value or produces a \textit{trap},
which is the result of executing the Wasm program.

%\begin{itemize}
%\item Borrowed implementation from the reference interpreter
%\item Handwritten parts?
%\end{itemize}
%./watsup spec/* --animate --sideconditions --interpreter
