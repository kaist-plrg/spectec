\subsection{Interpreter Backend}\label{sec:interp} %1p
\dslname supports the interpretation of Wasm programs by \textit{meta-level interpretation}.
By interpreting an \al program that denotes the Wasm semantics, with
a Wasm program as its input value, we can indirectly interpret the Wasm program.

This approach was previously used by the ESMeta framework~\cite{esmeta,jiset},
which extracts an executable semantics from the ECMAScript prose specification~\cite{ecmascript}.
ESMeta parses the structured English prose algorithms in the specification,
and translates them into its internal representation, \ires.
\ires has its own semantics, and thus can be executed with an interpreter implementation.
Executing the JavaScript semantics written in \ires allows an indirect execution of JavaScript programs.
Similarly, we first developed the \al interpreter in Ocaml based on the \al
semantics~\cite{il-tr}.  This interpreter allows us to execute the WebAssembly
semantics written in \al, enabling an indirect execution of WebAssembly
programs.

To initiate the interpretation process, SpecTec requires a valid input file,
which can be in one of three formats: `.wasm' representing a WebAssembly module
in binary format, `.wat' representing a WebAssembly module in text format, or
`.wast' denoting a WebAssembly module accompanied by assertions that verify if
invoking a Wasm function of the module produces the expectd result.  Currently,
SpecTec lacks the capability to parse these file formats directly.  Therefore,
we leverage the parser from the reference interpreter. Once the Wasm module is
parsed, it is converted into an \al value, which can be passed as an argument
to subsequent \al functions in the interpretation process.

There are two primary entry functions within the specification:
$\mathit{instantiate}$ and $\mathit{invoke}$. The $\mathit{instantiate}$
function is responsible for instantiating the given module obtained from
parsing the input file into a module instance.  This process involves
allocating the module into the store, initializing a table or memory from an
active segment, and executing the start function within the module. The role of
$\mathit{invoke}$ function is to invoke one of the functions within the Wasm
module. It takes the address of the function to be invoked and the
corresponding arguments as input, returning the result of the invocation as the
output of this \al function call.  Especially, this function is called when
handling assertions specified in `.wast' files, once per each assertion.

%\begin{itemize}
%\item Wasm text format, parser, ...
%\item Borrowed implementation from the reference interpreter
%\item Handwritten parts?
%\item How to execute a Wasm program
%\end{itemize}

%./watsup spec/* --animate --sideconditions --interpreter
% Interpretation of AL

