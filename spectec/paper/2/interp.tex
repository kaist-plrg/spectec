\subsection{Interpreter Backend}\label{sec:interp} %1p
\dslname supports the interpretation of Wasm programs by \textit{meta-level interpretation}.
By interpreting an \al program that denotes the Wasm semantics, with
a Wasm program as its input value, we can indirectly interpret the Wasm program.

This approach was previously used by the ESMeta framework~\cite{esmeta,jiset}.
ESMeta extracts two main components from the ECMAScript specification~\cite{ecmascript}:
parser and executable semantics.
The parser is obatined from the grammar section in the specification.
It accepts a JavaScript program as an input, and returns the parsed AST as an output.
The executable semantics is obtained from the structured English prose algorithms in the specification.
It is translated into the internal representation, \ires.
\ires has its own semantics, and thus can be executed with an interpreter implementation.
Executing the extracted semantics written in \ires, with the parsed AST as the input for the semantics,
allows an indirect execution of JavaScript programs.

SpecTec allows the indirect execution of Wasm programs in similar way.
The parser will parse the Wasm program into an AST, which will be used as an input argument of \al algorithms.
Unlike ESmeta, the Wasm parser is not yet automatically generated by SpecTec,
and for now, we rely on the parser from the reference interpreter.
The executable semantics written in \al is automatically extractd as described in Section~\ref{sec:dl2al}.
We developed the \al interpreter in Ocaml based on the \al semantics~\cite{il-tr},
which allows us to execute this extracted Wasm semantics.
Specifically, we can specify which \al algorithm to execute as a main entry of the extracted semantics.
Currently, the Wasm specification has two main entry algorithms:
$\mathit{instantiate}$ and $\mathit{invoke}$. 
The $\mathit{instantiate}$ algorithm instantiates the Wasm module into the module instance,
and the $\mathit{invoke}$ algorithm invokes one of the Wasm functions within the Wasm module.
\al interpreter will call these main algorithms while passing the parsed AST as one of their input arguments.
These algorithms in turn recursively call other algorithms,
including those corresponding to the semantics of individual Wasm instructions like `REF.IS\_NULL'.
Eventually, they return \al values as the result or a `trap' would occur.
This result constitutes the final result of executing the Wasm program,
encompassing both the instantiation of the Wasm module and the invocation of the Wasm function.

%\begin{itemize}
%\item Wasm text format, parser, ...
%\item Borrowed implementation from the reference interpreter
%\item Handwritten parts?
%\item How to execute a Wasm program
%\end{itemize}

%./watsup spec/* --animate --sideconditions --interpreter
% Interpretation of AL

