\subsection{Prose Backend}\label{sec:prose} %0.5p
\begin{figure}[t]
\centering
\includegraphics[width=.5\textwidth]{../img/genprose}
\vspace*{-1em}
\caption{Semantics of \inblue{\ensuremath{\mathsf{ref.is\_null}}} in a generated prose specification}
\label{fig:genprose}
\end{figure}

As \al is designed to resemble prose notation, 
generating an English prose specification from \al algorithms is a straightforward task.
Fig.~\ref{fig:genprose} shows the prose pseudocode of \inblue{\ensuremath{\mathsf{ref.is\_null}}} instruction,
generated from the specification in Fig.~\ref{fig:dsl},
which is very close to the original handwritten prose in Fig~\ref{fig:spec1}.
AL algorithms are rendered into a prose specification document through three steps.
First, the semantics described in \al is printed into English prose in reStructuredText markup.
For example, the third step of \inblue{\ensuremath{\mathsf{ref.is\_null}}} is printed as follows:
\begin{verbatim}
    3. If :math:`val` is of the case \
       :math:`\xref{exec/runtime}{syntax-ref}{\mathsf{ref{.}null}}`, then:
\end{verbatim}
Next, as in the LaTeX backend, prose in reStructuredText is spliced into a skeleton specification document.
Finally, the spliced document is processed by the Sphinx documentation tool~\cite{sphinx},
producing human-frieldnly formats like PDF and HTML.

Note that prose in reStructuredText is not simple plaintext,
but has inline math blocks as denoted by the \texttt{:math:} markup.
Expressions in math blocks are typesetted with LaTeX, thereby
making a clear distinction between mathematical notation and ordinary English phrases.
Furthermore, the prose backend embeds cross-references into the math blocks 
with \texttt{\textbackslash xref\{doc\}\{section\}\{text\}}.
This serves as a reference to a \texttt{section}
in some reStructuredText file \texttt{doc}, to be rendered as \texttt{text}.
As in the original specification document, 
\inblue{\ensuremath{\mathsf{ref{.}null}}} in the third step references its syntax production rule in a separate file.
This systematic insertion of references rules out
possibilities of missing, broken, or misplaced links when inserted manually.
