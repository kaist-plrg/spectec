%!TEX root = main.tex

\newpage
\section{Related Work}
\label{sec:related}

\textbf{Programming Language Frameworks}
Researchers have presented numerous frameworks to mechanize the
definitions of programming languages.
Once language definitions are mechanized, frameworks can provide various support
such as LaTeX typesetting and type checking of language definitions,
reference implementation in another language, and 
proof assistant code for theorem provers.
Ott~\cite{ott} allows language designers to specify the semantics in inference rules,
and generates code for Coq, HOL, and Isabelle/HOL.
It has been used for case studies like a large fragment of OCaml.
Spoofax~\cite{spoofax} supports agile development of textual
domain-specific languages with the Eclipse IDE support.
Skeleton~\cite{skeleton} specifies language semantics in big-step semantics
and constructs both concrete abstract interpretation.
PLTRedex~\cite{pltredex} describes language semantics in reduction rules,
and It has specified the semantics of Scheme~\cite{r6rs}.
The K framework~\cite{k} can generate various tools,
including interpreters, model checkers, and verifiers, from a language specification
written in its language K. It has specified the core semantics of real-world programming languages
such as C~\cite{kc}, Java~\cite{kjava}, Python~\cite{kpython}, and JavaScript~\cite{kjs}.
While the aforementioned frameworks aim to support general-purpose languages
with declarative semantics, ESMeta~\cite{esmeta} is designed to
support JavaScript with imperative semantics.
By devising a specific algorithmic language, \ires,
to specify the semantics described in the prose ECMAScript standard,
ESMeta can automatically generate diverse tools~\cite{jiset,jest,jstar,jsaver}.
While existing language frameworks are dedicated to a certain semantics style,
\dslname can handle both declarative and algorithmic styles.

\textbf{Wasm Semantics in Existing Language Frameworks}
The Wasm semantics have been mechanized in PLT-Redex and K.
Two PLT-Redex models specify a large core of Wasm~\cite{wasm-pldi17}:
wasm-redex~\cite{wasm-redex-asumu} and Wasm-Redex~\cite{wasm-redex-adam}.
The Wasm semantics defined by the collection of reduction rules is implemented
by the rewrite system of PLT-Redex.
Both models do not specify the Wasm module semantics, thus complete execution of
Wasm programs is not available.
KWasm~\cite{wasm-k} specifies the Wasm runtime semantics, including the module semantics,
in a single rewrite system in the K framework.
Since it does not model the entire Wasm semantics yet,
it fails for a number of tests in the official Wasm test suite.
While both PLT-Redex and K execute the Wasm semantics using their rewriting engines,
\dslname provides indirect interpretation over an algorithmic representation of the semantics.

\textbf{Wasm Semantics Mechanizations}
