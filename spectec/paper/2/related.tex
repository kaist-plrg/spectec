%!TEX root = main.tex

\section{Related Work}
\label{sec:related}

\textbf{Wasm Semantics in Existing Language Frameworks}
The closest effort to executing the Wasm semantics definition can be found in PLT-Redex and the K framework.
There are two PLT-Redex models of Wasm, wasm-redex~\cite{wasm-redex-asumu} and Wasm-Redex~\cite{wasm-redex-adam},
both based on the paper model of Wasm~\cite{wasm-pldi17}.
PLT-Redex actuates the rewrite system, defined by the collection of Wasm reduction rules.
Their shortcoming is that the module semantics is not specified,
thus complete execution of a Wasm program is not available.
KWasm~\cite{wasm-k} specifies Wasm utilizing the K framework.
Wasm runtime semantics, including the module semantics, is brought together 
into a single rewrite system in the K framework.
Yet, subtleties of the Wasm semantics is not entirely modeled in KWasm,
therefore it fails for a number of tests in the official Wasm test suite.
% Despite the rewrite systems in wasm-redex, Wasm-Redex, and KWasm are
% designed to resemble the formal notations in the specification,
% the inherent lack of expressibility in specification languages versus mathematical language 
% limits their executability.
% the statement above is quite wild ...
\dslname takes a different approach to executing Wasm semantics.
Instead of executing the rewrite system by itself,
it provides indirect interpretation over an algorithmic representation of the semantics,
The complete runtime semantics specified in \dslname is 
% technically, it is not complete ... (ex. SIMD)
systematically spelled out into an algorithmic form.
Thus, \dslname can thoroughly execute Wasm semantics,
from instantiating a module to invoking a function inside the module.
