\begin{abstract}
WebAssembly (Wasm) is a low-level bytecode language and virtual machine, intended as a compilation target for a wide range of programming languages,  which is seeing increasing adoption across
diverse ecosystems.  As a young technology, Wasm continues to evolve --- it reached version 2.0 last year and another major update is expected soon.

For a new feature to be standardised in Wasm, four key artefacts
must be presented: a formal (mathematical) specification of the feature,  an accompanying prose pseudocode description, an implementation in the official reference
interpreter, and a suite of unit tests.
This rigorous process helps to avoid errors in the design and implementation of new Wasm features, and Wasm's distinctive formal specification in particular has facilitated machine-checked proofs of various correctness properties for the language.
 However, manually crafting all of these artefacts requires expert knowledge combined with
repetitive and tedious labor, which is a burden on the language's standardization process and authoring of the specification.

This paper presents \emph{Wasm \dslname},
a technology to express the \emph{formal} specification of Wasm through a \emph{domain-specific language}. This DSL allows all of Wasm's currently handwritten specification artefacts to be error-checked and generated automatically from a single source of truth, and is designed to be easy to write, read, compare, and review.
We believe that \emph{Wasm \dslname}'s automation and meta-level error checking will significantly ease the current burden of the language's specification authors.
We demonstrate the current capabilities of Wasm \dslname by showcasing its
proficiency in generating various artefacts, and describe our work towards replacing the manually written
official Wasm specification document with specifications generated by Wasm \dslname.
\end{abstract}
